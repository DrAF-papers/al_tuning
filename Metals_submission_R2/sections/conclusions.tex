\label{conclusions}

This work shows that the Nelder-Mead search algorithm is an appropriate multidimensional search algorithm for the determination of the optimal dataset for improved simulation accuracy.  
It was capable of reducing the error of the simulated melt track depth and width of a set of processing parameters as measured by the sum of the error in the width and depth measurements (Equation \ref{eqn:response}) from over 600\% for a generic aluminum alloy to 9.1\%.
This was verified using a range of laser processing parameters to verify the effectiveness over a processing window.  This verified the results, using the same error measurement, the average error was improved by over 500\%
as shown in Figures \ref{fig:melt_track_val_baseline} and Figure \ref{fig:melt_track_val}, which used datasets from literature (Table \ref{tab:starting_mat_prop_complete}) and the optimized dataset (Table \ref{tab:7000_mat_prop_complete}) respectively.  
It was found that the values of the laser absorption at the liquidus temperature and the specific heat at 733\degree C, for the optimized dataset were triple that of the generic dataset.  Conversely, at 922\degree C, the generic dataset was triple that of the optimized dataset values.  The thermal conductivity of the optimized dataset was about double that of the generic dataset at 1491\degree C.  Lastly, the laser diameter crudely estimated was nearly double that of the optimized input dataset.
This methodology can be used to develop accurate simulations for any material that is not well published or to increase the accuracy of a simulation that utilizes approximations of first principles to increase its efficiency.