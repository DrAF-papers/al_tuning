\label{introAMsim}

\Acf{AM} is an emerging manufacturing technique which has the potential to revolutionize manufacturing.  To realize this revolution, it is necessary to be able to produce components reliably and to understand the process well enough to ensure that builds are consistent enough that the performance of the completed build can be guaranteed.  To do this, researchers and manufacturers have turned to mathematical modeling to understand the process \cite{neittaanmakiImpactScientificComputing2023}.
\added[comment={Added justification for need for more accurate thermal modeling\label{rev:need}}]{This body of work focuses solely on increasing the accuracy of thermal modeling of the process however the thermal cycling that a build experiences has an effect on the resulting microstructure \cite{qianSubrapidSolidificationStudy2020, weiMechanisticModelsAdditive2020, ansariSelectiveLaserMelting2022}, residual stress and distortion \cite{weiMechanisticModelsAdditive2020, ningAnalyticalModelingPart2020, promoppatumPartScaleEstimation2021}, and porosity of the build \cite{ningAnalyticalModelingPart2020a, wangPredictionLackoffusionPorosity2021, linProcessOptimizationDirected2020}.  If the accuracy of the underlying thermal modeling prediction can be improved then the subsequent properties can be more accurately modeled as well.}

The differences in simulation techniques can vary based on the desired response from the simulation and the underlying assumptions which were made during the development of the models.
One example of this can be seen when comparing the mathematical models \replaced{developed by Wang and Chen}{presented in} \cite{wangClosedLoopHighFidelitySimulation2021}, \added{Roy and Wodo} \cite{royDatadrivenModelingThermal2020}, and \added{Moges et al.} \cite{mogesHYBRIDMODELINGAPPROACH2020}.  They all attempt to model roughly the same aspect of the build but take very different approaches.
\added{Wang and Chen} \cite{wangClosedLoopHighFidelitySimulation2021} take a purely physics-based approach to the solution and works from first principle of the physical process being modeled \added{this is done for Ti-64 and was able to get an accuracy of between 3.6\% and 9.0\%}.
On the contrary, \added{Roy and Wodo} \cite{royDatadrivenModelingThermal2020} use a data driven model which used a breadth of data to develop a mathematical model which properly predicts the material behavior \added{this model is applied to the plastic \ac{FFF} process due to the volume of data required.  It would be possible to develop for a metal process, though could become cost prohibitive quickly.}
In between these models exists \added{Moges et al.} \cite{mogesHYBRIDMODELINGAPPROACH2020} which attempts to marry the two approaches and develop a physics-based model that uses data to improve accuracy \added{for Inconel 625 it was able to produce results with an average relative error of 7.58\%}.


The one unifying characteristic of these, and all, mathematical models, is the need for the inclusion of a dataset which defines the behavior of the material being investigated, this is colloquially referred to as the material properties.  These material properties can vary in literature and this variance in values can lead to a discrepancy in simulation results \cite{daryabeigiThermalPropertiesAccurate2011}.
\added{This need for input material properties extends to surrogate models which have been developed, such as \ac{ML} models, which will also require the inclusion of these material properties in order to properly predict the outcome \cite{zhuMachineLearningMetal2020, zobeiryPhysicsinformedMachineLearning2021, mengMachineLearningAdditive2020, wangMachineLearningAdditive2020}.}
\comment{Added notes of the material modeled and it's accuracy, \label{ref:modelMatAcc}}

Though variation exists in the literature, values can be found and used when the material is well characterized and published.  An example of a well published material is Ti-64 where it is easy to find literature which reports the material properties such as \replaced{Welsch et al.}{in} \cite{welschgerhard_1993}, \added{Boivineau et al.} \cite{boivineau_2006}, and \added{Fan and Liou} \cite{fan_2012}.  However, for materials which are not well \replaced{characterized}{understood} and published, such as specific aluminum alloys, it can be very challenging \deleted{though possible} to find some properties such as \added{reported by Lindberg} \cite{lundberg_material_1994} and \added{Leitner et al.} \cite{leitner_thermophysical_2017}.  There is a need to determine a complete dataset of the material properties, or at a minimum, develop the dataset which produces the most accurate simulation results. 
This can be accomplished by expending the necessary resources to measure the needed properties using advanced equipment.
This process can be expensive and time-consuming which has led to the development of material simulations which attempt to predict the material properties, such as JMatPro\cite{jmatpro}.  Though faster and cheaper than experimental results, they, like all simulations, have an error that is associated with them, \added{this is reported by Liu et al. \cite{liuStudyPredictionTensile2020}, Chen et al. \cite{chenMicrostructurePropertiesIronbased2023}, and Geng et al. \cite{gengDatadrivenMachineLearning2022}.}  Using these values alone can lead to unknown errors stacking up in the \ac{AM} models.
This problem also applies to materials where the tolerances for alloying elements is so wide that specimen of the same alloy can have different material properties.  

In order to develop a dataset which produces accurate \ac{AM} simulation results, \replaced{an}{a multi-objective} optimization \replaced{routine}{scheme} can be used as a search algorithm to determine the dataset which produces the most realistic results.  This can be used to develop a dataset for a new alloy along with for a specific batch of stock which has been procured from a supplier. 



This work will aim to address one of the current desires in metal \ac{AM} which is to be able to produce parts out of aluminum.  The desire is evident by the volume of effort being applied to aluminum \ac{AM} (\cite{qiHighStrengthLi2020, weissImprovedHighTemperatureAluminum2019, weissDevelopmentsAluminumScandiumCeramicAluminumScandiumCerium2019}).  The challenge associated with this stems from the wide range of alloys which have wildly varying material properties and are not well published.  One of the weldable high-strength alloys which has been targeted for metal \ac{AM} is 7050 \cite{singhAdditiveManufacturing4047}.  Though the material is widely available, temperature-dependent material properties are not readily available.  
\replaced[comment={Added explanation of plan for this body of work\label{ref:introPlan}}]{
	This work will perform an optimization routine on the material dataset for 7050 aluminum alloy to enable more accurate simulation results.  This will be done by validating the model on the well characterized Ti-64 material showing the model's effectiveness.  Then utilizing sister alloys, a generic material property dataset will be found from literature.  The search algorithm will then be applied to generate an optimal input properties dataset.  These optimized material properties will then be used with a range of laser parameters and compared experimentally to show their effectiveness over a processing window.
}{
	Therefore, this work will find material properties in literature which are an approximation of the 7050 aluminum, namely sister alloys with similar compositions, as a starting point for the search algorithm.  From that starting database, a search algorithm will be applied to develop a material dataset which produces more accurate \ac{AM} simulation results.
	}


