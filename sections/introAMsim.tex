\label{introAMsim}

\Acf{AM} is an emerging manufacturing technique which has the potential to revolutionize manufacturing.  In order to realize this revolution, it is necessary to be able to produce components reliably and to understand the process well enough to ensure that builds are consistent enough that the performance of the completed build can be guaranteed.  In order to do this, researchers and manufacturers have turned to mathematical modeling to understand the process \cite{wangClosedLoopHighFidelitySimulation2021}.

The differences in simulation techniques can vary based on the desired response from the simulation and the underlying assumptions which were made during the development of the models.
One example of this can be seen when comparing the mathematical models presented in \cite{wangClosedLoopHighFidelitySimulation2021}, \cite{royDatadrivenModelingThermal2020}, and \cite{mogesHYBRIDMODELINGAPPROACH2020}.  They all attempt to model roughly the same aspect of the build but take very different approaches.
\cite{wangClosedLoopHighFidelitySimulation2021} take a purely physics-based approach to the solution and works from first principle of the physical process being modeled.
On the contrary, \cite{royDatadrivenModelingThermal2020} is a data driven model which used a breadth of data to develop a mathematical model which properly predicts the material behavior. 
In between these models exists \cite{mogesHYBRIDMODELINGAPPROACH2020} which attempts to marry the two approaches and develop a physics-based model which uses data to improve accuracy.
The one unifying characteristic of these, and all, mathematical models, is the need for the inclusion of a dataset which defines the behavior of material being investigated, this is colloquially referred to as the material properties.  These material properties can vary in literature and this variance in values can lead to a discrepancy in simulation results \cite{Daryabeigi2011}.

Though variation exists in the literature, values can be found and used when the material is well characterized and published.  An example of a well published material is Ti-64 where it is easy to find literature which reports the material properties such as in \cite{welschgerhard_1993}, \cite{boivineau_2006}, and  \cite{fan_2012}.  However, for materials which are not well understood and published, such as specific aluminum alloys it can be very challenging though possible with \cite{lundberg_material_1994} and \cite{leitner_thermophysical_2017}, there is a need to determine the material properties, or at a minimum, develop the dataset which produces the most accurate simulation results. 
This can be accomplished by expending the necessary resources to measure the needed properties using advanced equipment.
This process can be expensive and time-consuming which has led to the development of material simulations which attempt to predict the material properties.  Though faster and cheaper than experimental results, they, like all simulations, have an error that is associated with them.  Using these values alone can lead to unknown error stack up in the \ac{AM} models.
This problem also applies to materials where the tolerances for alloying elements is so wide that specimen of the same alloy can have different material properties.  

In order to develop a dataset which produces accurate \ac{AM} simulation results, a multi-objective optimization scheme can be used as a search algorithm to determine the dataset which produces the most realistic results.  This can be used to develop a dataset for a new alloy along with for a specific batch of stock which has been procured from a supplier. 

This work will aim to address one of the current desires in metal \ac{AM} which is to be able to produce parts out of aluminum.  This desire is evident by the volume of effort being put into aluminum \ac{AM} (\cite{qiHighStrengthLi2020}, \cite{weissImprovedHighTemperatureAluminum2019}, \cite{weissDevelopmentsAluminumScandiumCeramicAluminumScandiumCerium2019}).  The challenge associated with this stems from the wide range of alloys which have wildly varying material properties which are not well published.  One of the weldable high strength alloys which has been targeted for metal \ac{AM} is 7050 \cite{singhAdditiveManufacturing4047}.  Though the material is widely available, temperature dependent material properties are not readily available.  Therefore, this work will find material properties in literature which are an approximation of the 7050 aluminum, namely sister alloys with similar compositions, as a starting point for the search algorithm and determine a material dataset which produces more accurate \ac{AM} simulation results. 


