\label{validation}

In order to ensure that the search algorithm results were valid across laser travel speeds and power levels a range of 8 other parameters were compared, the values used can be seen in Table \ref{tab:val_parameters}.
\begin{table}[!htb]
	\centering
	\caption{Validation processing parameters}
	\label{tab:val_parameters}
		\begin{tabular}{|c|c|c|} \hline 
			Exp. Id. & Scan speed (mm/min) & Laser Power (W) \\ \hline
			1 & 762 & 1000 \\ \hline  % 0
			2 & 762 & 1500 \\ \hline  % 1
			3 & 762 & 1250 \\ \hline  % 2
			4 & 1143 & 1250 \\ \hline % 3
			5 & 1143 & 1500 \\ \hline  % 5
			6 & 1524 & 1750 \\ \hline  % 6
			7 & 1524 & 1500 \\ \hline  % 7
			8 & 1524 & 2000 \\ \hline  % 8
		\end{tabular}
\end{table}

These speeds and powers were first completed with the literature determined values from Table \ref{tab:starting_mat_prop_complete} and the results can be seen in Figure \ref{fig:melt_track_val_baseline}.
\begin{figure}[!htb]\centering
	\begin{subfigure}[c]{0.45\textwidth}\centering
	\includegraphics[width=\textwidth]{melt_track_val_baseline_depth}
	\caption{Melt track depth}
	\label{fig:melt_track_val_baseline_depth}
	\end{subfigure}\hfill{}
		\begin{subfigure}[c]{0.45\textwidth}\centering
		\includegraphics[width=\textwidth]{melt_track_val_baseline_width}
		\caption{Melt track width}
		\label{fig:melt_track_val_baseline_width}
		\end{subfigure}
	\caption{Comparison of experimental and simulated results for validation points}
	\label{fig:melt_track_val_baseline}
\end{figure}
These results show that over the 9 initial parameter sets, when a melt track was developed, the average absolute value of the error in the depth was approximately 290\% and the average error in the width was approximately 265\%.  And to put into terms of the response variable of the search algorithm, the sum of the width and depth error, would be 555\% combined error.
Additionally, run 1 was unable to develop a melt pool, which is contrary to the experiments where all the parameter sets had a stable melt track.  These results corroborate the results from Figure \ref{fig:response_complete} which showed that for the parameter set used for tuning initial dataset of material properties found in literature is wholly inadequate for simulating the process at hand. 

In contrast to these results, the material dataset which was found in Table \ref{tab:7000_mat_prop_complete} was used to simulate each parameter set, and the results can be seen in Figure \ref{fig:melt_track_val}.
\begin{figure}[!htb]\centering
	\begin{subfigure}[c]{0.45\textwidth}\centering
	\includegraphics[width=\textwidth]{melt_track_val_depth}
	\caption{Melt track depth}
	\label{fig:melt_track_val_depth}
	\end{subfigure}\hfill{}
		\begin{subfigure}[c]{0.45\textwidth}\centering
		\includegraphics[width=\textwidth]{melt_track_val_width}
		\caption{Melt track width}
		\label{fig:melt_track_val_width}
		\end{subfigure}
	\caption{Comparison of experimental and simulated results for validation points}
	\label{fig:melt_track_val}
\end{figure}
These results show the average error in the width was approximately 17\% and the average error in the depth was approximately 5\%, which creates a combined error of only 22\%.  These results show that the optimized dataset is better at predicting the combined error of the simulation by over 500\%.  This results in a simulation which can be leveraged more intensely during the process development and build qualification process.  